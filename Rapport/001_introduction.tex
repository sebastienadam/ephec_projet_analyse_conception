\chapter{Introduction}

\section{Énoncé}

En tant qu'intermédiaire entre des propriétaires de biens immobiliers et d'éventuels locataires ou acheteur, une agence immobilière propose les différents biens suivants:
\begin{itemize}
  \item à louer ou à acheter: des biens immobiliers d'habitation (studios, appartements, maisons) et des biens immobiliers commerciaux (entrepôts, emplacements pour bureaux ou commerce);
  \item à acheter uniquement: des terrains à bâtir.
\end{itemize}

\subsubsection{Les classes standards}

De manière à pouvoir servir efficacement, à la fois, les propriétaires (offrants) et les clients (demandeurs), elle s'est définie un certain nombre de \og{}classes standard\fg{} de biens immobiliers; par exemple: la classe des terrains à bâtir de 10 ares et de moins de 30.000 euros, la classe des maisons d'habitation à louer comprenant au minimum deux chambres et dont le loyer mensuel serait inférieur à 1.500 euros, la classe des maisons d'habitation à acheter comprenant au minimum trois chambre et dont le prix demandé serait inférieur à 250.000 euros.

Une classe standard est identifiée par un code de classe et caractérisée par le type de biens immobiliers qui la composent (maison, appartements, studio, entrepôt, emplacement, terrain), leur mode d'offre (à louer, à acheter), un prix maximum et une superficie minimum.

Dans le cas d'appartement à louer, le prix maximum correspond à un prix mensuel maximum de location; pour les biens à acheter, il correspond à un prix maximum d'achat.

Dans le cas d'appartement ou maison, la superficie minimale correspond à un nombre de chambres; dans le cas d'immeubles commerciaux ou de studios, à une superficie exprimée en m²; dans le cas d'un terrain à bâtir, à une superficie exprimée en ares.

Pour exercer son activité, l'agence immobilière gère les informations suivantes:
\begin{itemize}
  \item Pour chaque propriétaire: son nom, son adresse (rue et numéro, code postal, localité), deux numéros de téléphone (privé et travail) et les heures de présence à ces numéros, ainsi que la liste des biens qu'elle est chargée de négocier pour eux.
  \item Pour tout bien immobilier: son statut (disponible, loué ou acheté), la classe standard à laquelle il appartient, la date à laquelle le bien lui a été soumis, sa localisation (rue et numéro, code postal et localité), la date de mise en disposition, le revenu cadastral, la liste des clients qui ont demandé à visiter ainsi que, les dates et heures de chaque visite, et les coordonnées de la personne de l'agence responsable de celle-ci. Enfin, s'il y a lieu, les coordonnées du client acquéreur (nom, adresse, téléphone), les prix et date effectifs d'achat où de location et le numéro de référence du contrat.
  \item Pour tout bien immobilier sauf terrain: une description du contenu en termes de nombre et nature des pièces, type de chauffage, etc. (texte libre).
  \item Pour tout bien immobilier à louer: le montant de la caution locative, le loyer mensuel, le montant mensuel des charges, le type de bail, la \og{}garniture\fg{} (meublé, non meublé).
  \item Pour tout bien immobilier à acheter: le prix d'achat demandé.
  \item Pour tout bien immobilier à acheter, sauf terrain: l'état (à restaurer, correct, impeccable).
  \item Pour tout bien de type \og{}maison\fg{} ou \og{}appartement\fg{}: le nombre de chambres, le nombre de garages, la présence ou non d'une cuisine équipée, la superficie du jardin éventuel.
  \item Pour tout bien de type \og{}maison\fg{}: le nombre d'étages.
  \item Pour tout bien de type \og{}appartement\fg{} ou \og{}studio\fg{}: l'étage auquel il est localisé et la présence ou non d'ascenseur.
  \item Pour tout bien de type \og{}terrain\fg{}: sa superficie.
  \item Pour tout bien de type \og{}emplacement pour bureaux ou commerce\fg{}: sa superficie et le nombre de pièces le composant.
  \item Pour un client: son nom, son adresse, son numéro de téléphone, les types de biens qu'il recherche, c'est-à-dire la liste des classes standards qui correspondent aux types de biens qui l'intéressent.
\end{itemize}

Pour l'agence immobilière, un client correspond à toute personne s'adressant à ses services pour louer ou acheter un bien immobilier. Il devient acquéreur s'il loue ou achète un bien immobilier par son intermédiaire.

Un propriétaire est une personne qui possède des biens immobiliers et s'adresse à l'agence pour les présenter à ses clients.

Un propriétaire peut posséder plusieurs biens immobiliers.

Un bien immobilier ne peut être la propriété que d'un seul propriétaire.

Un bien immobilier est soit à louer, à acheter. Un numéro permet de l'identifier parmi tous les biens immobiliers appartenant à la même classe.

Un bien immobilier appartient toujours à une et une seule classe standard.

Une classe standard peut ne contenir aucun bien immobilier.

Un client peut être intéressé par plusieurs classes de biens.

Un client est identifié par un numéro attribué par compostage. Il peut visiter plusieurs fois le même bien accompagné d'un responsable différent.

Lorsqu'un client se présente à l'agence immobilière, il soumet sa demande au service d'enregistrement des demandes. Il indique à l'employé le type de bien qu'il désire, s'il désire louer ou acheter, ses contraintes budgétaires ainsi que les principales caractéristiques des biens qui peuvent l'intéresser.

L'employé enregistre sa demande, c'est-à-dire qu'il associe au client les différentes classes standard correspondant au mieux à la description qui lui a été faite et il enregistre les informations relatives au client si celui n'est pas connu de l'agence.

La terminaison de l'enregistrement de la demande déclenche automatiquement, pour chaque classe standard correspondant à la demande du client, l'impression de la liste des biens encore disponibles appartenant à cette classe (cette liste reprend la localisation du bien, le prix demandé et les informations relatives à la superficie).

Le client examine les différentes listes et élimine de suite les biens immobiliers qui ne l'intéressent pas. S'il reste des biens susceptibles de l'intéresser, il s'adresse au service des visites.

Pour chacun des biens retenu par le client, un employé consulte, à l'aide d'un terminal, les informations s'y rapportant afin de fournir des renseignements plus précis tandis que son collègue recherche la photo correspondant dans le fichier. Grâce aux renseignements supplémentaires et à la photo, le client peut se faire une opinion sur le bien. L'employé enregistre alors l'accord ou le désaccord du client.

La terminaison de la consultation de tous les biens déclenche automatiquement, pour chaque bien retenu, l'affichage des dates et heures de visites déjà planifiées pour les autres clients intéressés par ce bien. En fonction de ces informations l'employé et le client conviennent ensemble d'une date et d'une heure de visite que l'employé enregistre. Il enregistre également le nom de la personne responsable de cette visite.

De manière à fournir régulièrement des informations pertinentes au service de prospection, le service d'enregistrement des demandes produit un document statistique récapitulant les différents types de demandes. Cet état statistique est généré en fin de semaine ou dès que 100 demandes sont enregistrées.

\section{Rapport}

Pour la réalisation de ce travail, j'ai d'abord commencé par les cas d'utilisation, le diagramme d'entités-associations et le diagramme de classes. Cela m'a permis d'appréhender le projet dans son ensemble. De plus, cela m'a également donné le recul nécessaire afin de décider quelle cas d'utilisation j'allais choisir pour le développer.

Après avoir réalisé les trois diagrammes précédemment cités, j'ai choisi le cas d'utilisation \selectedusecase{} pour le développer. J'ai réalisé son diagramme d'activité et son diagramme de séquence.